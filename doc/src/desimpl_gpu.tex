As no known open source design of a modern embedded graphics accelerator was available, we had to deign our own version of what we thought was necessary for creating an embedded graphics accelerator. We decided to approach a more retro-esque design and base it more on the likes of an Atari ST or a Commodore Amiga as discussed in \cref{subsec:blitter}. The basic functionality of our hardware accelerator is to draw lines with any two given coordinates, drawing circles with a given center and a radius, drawing a given bitmap font, and lastly bit-blitting a sprite on the screen.

For the line and circle drawing algorithms we want to use algorithms of the Bresenham family. These will be discussed in \cref{subsec:des_bresenham}. As the bit-blitter from the Commodore Amiga seemed for us to complex for our needs, we decided do design this in more of the approach of an Atari ST, as this will be discussed in \cref{subsec:des_blit}.