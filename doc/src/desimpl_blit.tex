\subsubsection*{Design}
\begin{itemize}
	\item two copy system, the first one to copy a font and the second one is to copy a sprite
	\item both of them has 8x8 pixel
	\item the difference between them are that the font has one colour and a alpha mask, the sprite can have on every pixel a different colour and has a alpha mask
	
\end{itemize}
\subsubsection*{Implementation}
\begin{itemize}
	\item for the font copy a object is made
	\item the font copy object has a state machine
	\item the first state is the idle state
	\item when the start signal is high, it set the counter to zero and goes to the copy state
	\item here is a big switch case, every value in the alpha mask has its one case
	\item when a bit in the alpha mask is high the write signal is high
	\item when the counter reaches 63 it goes to the idle state
	\item to calculate the address the first three bits in the counter variable are the x coordinate and last three bits is for the y coordinate
	\item the sprite are copy directly to the frame buffer.
	\item it has also a big switch case
	\item after it gets the alpha mask it write the value from the sprite if the corospending alpha bit is set  
\end{itemize}