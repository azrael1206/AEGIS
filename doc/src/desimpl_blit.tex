\subsubsection*{Design}
As we want our bit blitter to be nothing more than the memory-to-memory blocking in the Atari ST. We are going to split this process up into two parts. One for the font drawing and one for the sprite drawing, as they have minor differences in how the data looks like.

Both of the will consist of eight by eight pixel information that has to be copied, the difference between them is that the font only consists of an alpha mask, the area which will be drawn, and one color for the letter to be drawn in. The sprite on the other hand has beside the alpha mask, the sprite data that needs to be drawn. This consists of an eight by eight grid of color information. 

\subsubsection*{Implementation}
A separate object is made to copy a letter from the font into the frame buffer. In its state machine we differentiate between the idle state and the copy state, seen in \cref{fontfsm}. When the idle state gets triggered it waits till the start flag is set, similar as in \cref{subsec:des_bresenham}. Once it is triggered to draw it sets a counter to zero and switches into the copy state.
\begin{lstlisting}[language=scala, caption={States of the Font Blitter}, label=fontfsm]
val idle = new State with EntryPoint
val copy = new State
\end{lstlisting}
In the copy state we evaluate each pixel of the alpha mask in a big switch-case. When the value of the pixel is set to high we write the color information for the corresponding pixel. When the whole alpha mask is done, for our case when the counter reaches 63, we go back into the idle state.
to know to which pixel to draw to, the first three bits act as a part of the x-coordinate and the last three bits count for y-coordinate.
\begin{lstlisting}[language=scala, caption={Copying the sprite color information into the frame buffer}, label={blitbuffer}]
for (i <- 4 to 67) {
	is(i) {
		vga.io.wData(0) := buffer(10 downto 0).resized
		vga.io.wData(1) := buffer(21 downto 11).resized
		vga.io.wData(2) := buffer(31 downto 12).resized
		vga.io.wAddress := (!switchVGA 
				  			## (storeVals1(1) + temp(5 downto 3)) 
				  			## (storeVals1(0) + temp(2 downto 0))).asUInt
		vga.io.wValid   := alpha(i - 4)
		counter := counter + 1
		valid := True
		goto(readRam)
	}
}
\end{lstlisting}
The sprite blitter is part of our core, which is discussed in \cref{subsec:core}. The way we have to copy the data functions similarly to how it is done in the font copying. We have to check if the alpha mask is set at a certain pixel. If it is we can draw the pixel of the sprite, this can be seen in \cref{blitbuffer}. Here we also use a switch-case to check the corresponding pixels.
