\subsubsection*{Design}
The needs for the implementation were clear from the start of the project. We knew that we are going to need some sort of output to a screen and a framebuffer for an image to be drawn to.

The design of the VGA interface was held flexible, as we wanted the option not only to exchange it with a different interface, but also that we are able to instantiate it with different resolutions. If the FPGA would allow it, a resolution of up to 1080p would be theoretically possible, as it generates its clock from a master clock that is fed in.

The framebuffer is designed to hold two frames at the same time, meaning it is a double buffer. This is something taken from GPU design, as this eliminates the chance of screen tear. This happens when one would want to draw into the image as it is being projected to the screen. Additionally the framebuffer should be instantiated with the maximal size of a frame and its depth of color.

To have the VGA interface and our graphics accelerator access it at the same time, we also decided to give it a true dual port design. As one of the ports is for the VGA interface to read from, while the other port is the the graphics accelerator to read from and write to.
\subsubsection*{Implementation}
\begin{itemize}
	\item the framebuffer object as two read and one write port. 
	\item instanciate with the data it becomes from the VGAConfig
	\item it calculates the size of the mem automaticly. 
	\item first it looks if the resolution is a size of power of two  //code
	\item after that it multiplcate the x and y resolution after that it muliplikate it with two //code
	\item first the vga generate a slow clock area with the frequenc in the VGAConfig
	\item after that it set up some signals and register
	\item the io are connected to the corropending signals
	\item after that some checks are made it y and x counter reaches some special values like end of the line and end of the frame an d picture draw
	\item the buffer control controlls witch buffer the vga can read
	\item the buffer control connects the buffer with the vga and the MCP and here is the logic for the pixel doupler
\end{itemize}