\subsubsection*{Design}

\begin{itemize}
	\item the framebuffer is a true dual port design
	\item one port is for the vga signal generator, and the second one is for the gpu to read and to write
	\item is a double buffer, for that the framebuffer is divided in two half, one half acutal frame and the second half is the new frame
	\item framebuffer is instanciate with the max size of the frame and depth of the colour  
	\item the vga interface can be instanciate wiht the resolution
	\item the theoretical max resolution is 1080p
	\item it generates is clock from a master clock
\end{itemize}
\subsubsection*{Implementation}
\begin{itemize}
	\item the framebuffer object as two read and one write port. 
	\item instanciate with the data it becomes from the VGAConfig
	\item it calculates the size of the mem automaticly. 
	\item first it looks if the resolution is a size of power of two  //code
	\item after that it multiplcate the x and y resolution after that it muliplikate it with two //code
	\item first the vga generate a slow clock area with the frequenc in the VGAConfig
	\item after that it set up some signals and register
	\item the io are connected to the corropending signals
	\item after that some checks are made it y and x counter reaches some special values like end of the line and end of the frame an d picture draw
	\item the buffer control controlls witch buffer the vga can read
	\item the buffer control connects the buffer with the vga and the MCP and here is the logic for the pixel doupler
\end{itemize}