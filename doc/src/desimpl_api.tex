\subsubsection*{Design}

Before the implementation of the application programming interface (in short API), we had to decide if the user would have to code most of the features from scratch or use the functions we provide for the API.
We decided to choose a mix of both approaches, as we are only giving access to our basic functions of AEGIS. A function of the API should look as follows:
\begin{lstlisting}[language=C, caption={The Circle Drawing Function}, label=CircDraw]
uint32_t ae_draw_circle(uint32_t x, uint32_t y, uint32_t r, Color col);
\end{lstlisting}
The return type of the function indicates, if the function executed successfully (return value \(0\)) or executed with an error (return value greater than \(0\)). To decide if an error occurred or not we would have to read a register flag of the graphics accelerator. A special parameter of the function is the color value. As this should be a 32-bit value, we decided that the user should give us a color as three 32-bit values for each separate color channel as seen in \cref{colStruct}.
\begin{lstlisting}[language=C, caption={The Color Stuct}, label=colStruct]
typedef struct {
	uint32_t red;
	uint32_t green;
	uint32_t blue;
} Color;
\end{lstlisting}

To access the function on the graphics accelerator itself, we plan to use two pointers. One pointer should be used to access the address of the function, while the other one should be used to hold the actual data. 

\subsubsection*{Implementation}

During the previously explained design process, we had to work out how implement the API so that a user can work with all the functionality our graphics accelerator offers. To hold the data and access the functionality we had to use a pair of pointers. As seen in \cref{pointPair} the volatile pointer gives the user access to the function while the regular pointer holds the value for the call.
\begin{lstlisting}[language=C, caption={Pair of pointers}, label=pointPair]
volatile uint32_t* circle_addr;
Circle* circle_val;
\end{lstlisting}
To access the functionality of the hardware accelerator, we did not want the users to have to work with the raw pointers, as for that we created separate functions that send the call to the hardware. As it is seen in \cref{funcStruc} the basic structure of the is as follows. First we organize the given data in the corresponding value structure, then we pass the values to the address pointer. Afterwards we return zero to tell the user that the function ended properly. The return value of the function is a mix of legacy code and future work. This will be discussed in \cref{sec:future}.
\begin{lstlisting}[language=C, caption={Structure of the functions}, label=funcStruc]
uint32_t ae_draw_circle(uint32_t x, uint32_t y, uint32_t r, Color col) {
	circle_val->x1 = x;
	circle_val->y1 = y;
	circle_val->r = r;
	circle_val->col = COLOR(col.red, col.green, col.blue); 

	*circle_addr = (uint32_t) circle_val;

	return 0;
}
\end{lstlisting}
The only outlier from this pattern are the functions to initialize and de-initialize the pointers and the function to switch the frame buffers (see \cref{switchBuff}). As this does not have an underlying structure to hold any data, we opted for just accessing the address and giving it a random value. In this case the value of one. As this could not potentially go wrong we decided to omit the return value of the function.
\begin{lstlisting}[language=C, caption={Switching the Buffer}, label=switchBuff]
void ae_switch_buffer(){
	*switch_addr = 1;
}
\end{lstlisting}