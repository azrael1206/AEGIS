\subsubsection*{Atari ST and Amiga Blitter} 
The functionality of a Bit Block Transfer Processor or Bit Blitter can be simply described as moving bit aligned data from a source to a destination with certain given logical operations. Not only could a Bit Blitter move sprites, but had an array of other features like filling an area with a given color or a pattern, transforming texts or rotating objects~\cite{atari1987blit}. Although its main purpose was a general memory-to-memory block copying.

Throughout the mid-1980s to the early 1990s several iterations of such a co-processor were created. The two major Bit Blitters known today were in the Atari ST and in the Commodore Amiga~\cite{data1988amiga}. Their basic functionality was virtually the same, they differentiated in one major aspect though. Their approach to moving data was different.

\subsubsection*{Moving Data}
In general terms the Bit Blit Processor took two memory blocks, usually a sprite and a mask~\cite{data1988amiga}. It combined them through a logical operation e.g a XOR, and the moved the result into the destination block.

The Atari STs approach for that type of operation was rather straight forward. It took as previously described, the result of the two sources with a logical operation and transferred it bit by bit. Meanwhile the Commodore Amiga took this design and expanded upon it~\cite{amigaBlitter}. It offered several modes of operation, among others a block mode and a line mode. Additionally the Amiga Blitter is does not operate on a bit level, but copies the data 16-bit word wise meaning it is a word Blitter.

Taken that into the account, if we have a screen with the resolution of \(320 \times 200\) with a total of 16 displayable colors and four bit-planes. The Amiga Blitter would have to do 20 operations per row on it as it consists of \(200\) rows of \(40\) bytes of data   

