\subsubsection*{Design}
\begin{itemize}
	\item 
\end{itemize}
\subsubsection*{Implementation}
\begin{itemize}
	\item both the line and circle works with a state machine
	\item the interface of the line algorithm are two coordinates the first one is the start coordninate and the second one is the end coordinate. The coordinate are a vector of two UInt values. After that comes the start and ready signal. At the end the address where the pixel is written and a signal to write
	\item the first state is the idle state here it waits until the start signal is high
	\item on the line algorithm it after the start signal come it calculate needed variables
	\item the next to states in the line algorithm calculate also needed variables, why in other states because of the depencies of the variables
	\item after that in the line algorithm comes the main part where the line is drawn, is the line at the right postition it goto the idle state
	\item the cirlce interface has simalarity to the line interface. the only difference between them is that the scond cooordinate in the line algorithm is a radius in the cirle algorithm
	\item the circel alogrithm after the start signal comes, it set the some variables
	\item after the idle state it comes four draw state.
	\item four states because we can save only one pixel per cycle
	\item two calculate state comes after the drawing states.  
	\item here are the same reason for two states like the line algorithm, because of depencies of the variables
\end{itemize}