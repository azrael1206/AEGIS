\documentclass[a4paper]{article}

\usepackage{blindtext} % Package to generate dummy text throughout this template 

\usepackage[sc]{mathpazo} % Use the Palatino font
\usepackage[utf8]{inputenc}
\usepackage[T1]{fontenc} % Use 8-bit encoding that has 256 glyphs
\linespread{1.05} % Line spacing - Palatino needs more space between lines
\usepackage{microtype} % Slightly tweak font spacing for aesthetics

\usepackage[american]{babel} % Language hyphenation and typographical rules

\usepackage[hmarginratio=1:1,top=32mm]{geometry} % Document margins
\usepackage[hang, small,labelfont=bf,up,textfont=it,up]{caption} % Custom captions under/above floats in tables or figures
\usepackage{booktabs} % Horizontal rules in tables

\usepackage[backend=biber,style=ieee]{biblatex}
\usepackage{csquotes}
\bibliography{bibliography} 

\usepackage{mathtools}

%%%%%%%%%%%%%%%%%%%%%%%%%%%%%%%%%%%%%%%%%%%%%%%%%%%%%%%%%%%%%%%%%%%%%%%%%%%%%%%
% Bilder
%%%%%%%%%%%%%%%%%%%%%%%%%%%%%%%%%%%%%%%%%%%%%%%%%%%%%%%%%%%%%%%%%%%%%%%%%%%%%%%
\usepackage{graphicx}
\usepackage{wrapfig}
\usepackage{float}
\graphicspath{{./resources/img/}}

%\usepackage{lettrine} % The lettrine is the first enlarged letter at the beginning of the text

\usepackage{enumitem} % Customized lists
\setlist[itemize]{noitemsep} % Make itemize lists more compact

%%%%%%%%%%%%%%%%%%%%%%%%%%%%%%%%%%%%%%%%%%%%%%%%%%%%%%%%%%%%%%%%%%%%%%%%%%%%%%%
% Listings
%%%%%%%%%%%%%%%%%%%%%%%%%%%%%%%%%%%%%%%%%%%%%%%%%%%%%%%%%%%%%%%%%%%%%%%%%%%%%%%
\usepackage{listings}

\usepackage{xcolor}
\lstset{
	numbers=left,
	numberstyle=\tiny,
	numbersep=5pt,
	breaklines=true,
	showstringspaces=false,
	frame=single ,
	xleftmargin=15pt,
	xrightmargin=15pt,
	basicstyle=\ttfamily \footnotesize,
	stepnumber=2,
	captionpos=b,
	language=C,   %Sprache Festelegen
	floatplacement=H,
	tabsize=2,
	keepspaces=true
	%	keywordstyle=\color{blue},          % keyword style
	%	commentstyle=\color{dkgreen},       % comment style
	%	stringstyle=\color{mauve}         % string literal style
}


\lstdefinelanguage{scala}{
	morekeywords={abstract,case,catch,class,def,%
		do,else,extends,false,final,finally,%
		for,if,implicit,import,match,mixin,%
		new,null,object,override,package,%
		private,protected,requires,return,sealed,%
		super,this,throw,trait,true,try,%
		type,val,var,while,with,yield},
	otherkeywords={=>,<-,<\%,<:,>:,\#,@},
	sensitive=true,
	morecomment=[l]{//},
	morecomment=[n]{/*}{*/},
	morestring=[b]",
	morestring=[b]',
	morestring=[b]"""
}


\lstset{inputpath=./resources/code}


\usepackage{abstract} % Allows abstract customization
\renewcommand{\abstractnamefont}{\normalfont\bfseries} % Set the "Abstract" text to bold
\renewcommand{\abstracttextfont}{\normalfont\small\itshape} % Set the abstract itself to small italic text

\usepackage{titlesec} % Allows customization of titles
\titleformat{\section}[block]{\Large\scshape}{\thesection.}{1em}{} % Change the look of the section titles
\titleformat{\subsection}[block]{\large\scshape}{\thesubsection.}{1em}{} % Change the look of the section titles

\usepackage{fancyhdr} % Headers and footers
\pagestyle{fancy} % All pages have headers and footers
\fancyhead{} % Blank out the default header
\fancyfoot{} % Blank out the default footer
\fancyhead[C]{} % Custom header text
\fancyfoot[RO,LE]{\thepage} % Custom footer text

\usepackage{titling} % Customizing the title section

\usepackage[hidelinks]{hyperref} % For hyperlinks in the PDF
\usepackage{cleveref}

%----------------------------------------------------------------------------------------
%	TITLE SECTION
%----------------------------------------------------------------------------------------

\setlength{\droptitle}{-4\baselineskip} % Move the title up

\pretitle{\begin{center}\Huge\bfseries} % Article title formatting
	\posttitle{\end{center}} % Article title closing formatting
\title{An Architecture for Embedded Graphical Interfaces} % Article title

\author{%
	\textsc{Brendan Christy} \and \textsc{Ingo Braun} \and\\[-2ex]
	\vspace{1pt} \footnotesize Department of Computer Science, University of Applied Sciences RheinMain, Germany \\ 
	\vspace{1pt} \footnotesize \texttt{\{brendan.b.christy, ingo.braun\}@student.hs-rm.de}
}

\date{\today} % Leave empty to omit a date

\renewcommand{\maketitlehookd}{%
	\begin{abstract}
		\noindent \blindtext
	\end{abstract}
}

%----------------------------------------------------------------------------------------

\begin{document}
	
	% Print the title
	\maketitle
	
	%----------------------------------------------------------------------------------------
	%	ARTICLE CONTENTS
	%----------------------------------------------------------------------------------------
	\section{Introduction}
	
	\section{Prerequisites}
	
	\subsection{GPU}
	\begin{itemize}
	\item Basic functionality of modern GPUs \cite{kilgariff2005geforce} have not changed over the years \cite{nvidia2018turing}
	\item Based on a graphics pipeline:
	\item Vertex Processing: Takes  three dimensional vertices and transforms it into screen space (what means screen space), each vertex independently
	\item Primitive Processing: Takes primitives, in this case a group of vertices that belong together and form a triangle, clips and culls them (what does that mean)
	\item Rasterization: Primitives are rasterized into so called pixel fragments (a group of undefined pixels without color information)
	\item Fragment processing: Afterwards these pixel fragments are then shaded to compute color at each pixel
	\item Pixel operation: Fragments are bleneded into the frame at their pixel locations (z-buffer determines visibility)
	\item as this is needed for each unit(vertex, primitive, fragment and pixel) independently, this is the of one core.
	\item A modern GPU can consist now at up to 2944 cores that all act in somewhat of the previously described functionality
\end{itemize}

	
	\subsection{Bresenham Algorithm}
	The original Bresenham Algorithms was developed by J. Bresenham in 1965~\cite{Bresenham65Line}. The target was to build a algorithm to draw a line between to coordinates on a pixel grid. The family of Bresenham Algorithms that exist, e.g. to draw a circle are developed by others. In the following section, the Bresenham Line and Circle algorithm will be explained.
\subsubsection*{Line Drawing}
The original line drawing algorithm uses only additions, subtractions, and a multiplication with the constant value of two. In a Cartesian coordinate system it exist eight variation of a gradient from a line, called octanes. The algorithm uses only one octants. It archived it by transforming the coordinates.

\begin{algorithm}[H]
	\SetAlgoLined
	\KwIn{\(x_0,~ y_0,~ x_1,~ y_1\)}
	\(\varDelta_x \gets \abs{x_1 - x_0}\)\;
	\(\varDelta_y \gets \abs{y_1 - y_0}\)\;
	\(\text{step}_x \gets x_0 < x_1~ ? ~1 ~:~ -1 \)\;
	\(\text{step}_y \gets y_0 < y_1~ ?~ 1 ~:~ -1\)\;
	err \( \gets \varDelta_x + \varDelta_y \)	
	\While{\(x_0 \neq x_1\) and \(y_0 \neq y_1\)}{
		draw Pixel at \(x_0 | y_0\) \;
		err2 \(\gets 2 \times err\)\;
		\If{err2 \(> \varDelta_y\)}{
			err \(\gets \text{err} + \varDelta_y\)\;
			\(x_0 \gets x_0 + \text{step}_x\)\;
		}
		\If{err2 \(< d_x\)}{
			err \(\gets \text{err} + \varDelta_x\)\;
			\(y_0 \gets y_0 + \text{step}_y\)\;
		}	
	}
	\caption{Bresenham Line Drawing Algorithm}
	\label{alg:bresline}
\end{algorithm}

The first step is to set the start coordinate. After that an error variable is set. The error variable is for the decision, which pixel is set next. Is value of the error variable smaller than zero it goes one step in the x direction, is the value bigger equal to zero it goes one step in the y and one step in the x direction. It ends when the new x and y coordinate equals the end x and y coordinate.
 
\subsubsection*{Circle Drawing}
The circle algorithm function are similar to the line algorithm but it has some differences. One difference is, that it has a circle radius and the coordinate of the middle point of the circle.
	
	\subsection{Bit Blitter}
	\subsubsection*{Atari ST and Amiga Blitter} 
The functionality of a Bit Block Transfer Processor or Bit Blitter can be simply described as moving bit aligned data from a source to a destination with certain given logical operations. Not only could a Bit Blitter move sprites, but had an array of other features like filling an area with a given color or a pattern, transforming texts or rotating objects~\cite{atari1987blit}. Although its main purpose was a general memory-to-memory block copying.

Throughout the mid-1980s to the early 1990s several iterations of such a co-processor were created. The two major Bit Blitters known today were in the Atari ST and in the Commodore Amiga~\cite{data1988amiga}. Their basic functionality was virtually the same, they differentiated in one major aspect though. Their approach to moving data was different.

\subsubsection*{Moving Data}
In general terms the Bit Blit Processor took two memory blocks, usually a sprite and a mask~\cite{data1988amiga}. It combined them through a logical operation e.g a XOR, and the moved the result into the destination block.

The Atari STs approach for that type of operation was rather straight forward. It took as previously described, the result of the two sources with a logical operation and transferred it bit by bit. Meanwhile the Commodore Amiga took this design and expanded upon it~\cite{amigaBlitter}. It offered several modes of operation, among others a block mode and a line mode. Additionally the Amiga Blitter is does not operate on a bit level, but copies the data 16-bit word wise meaning it is a word Blitter.

Taken that into the account, if we have a screen with the resolution of \(320 \times 200\) with a total of 16 displayable colors and four bit-planes. The Amiga Blitter would have to do 20 operations per row on it as it consists of \(200\) rows of \(40\) bytes of data   


	
	\section{Implementation}
	
	\section{Conclusion}
	
	\section{Future Work}
	
	\printbibliography
	
	
\end{document}